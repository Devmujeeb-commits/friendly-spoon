
\documentclass[12pt]{article}
\usepackage{amsmath, amssymb}
\usepackage{geometry}
\geometry{margin=1in}
\usepackage{lmodern}
\usepackage{enumitem}

\title{Collection of Proofs in Real Analysis \\ \large Proof 1: On the Indeterminacy of \texorpdfstring{$\frac{0}{0}$}{0/0}}
\author{Group 6}
\date{}

\begin{document}



\section*{Group Members}
\begin{enumerate}[label=\arabic*.]
    \item \textbf{UTHMAN Abdulmujeeb Kolawole} \hfill 243984
    \item \textbf{EKEATA John Eromonsele} \hfill 243957
    \item ADEKUNLE Joshua Oladipupo \hfill 243934
    \item GANIYU Isa Oladipupo \hfill 251021
\end{enumerate}

\section*{Theorem}
\textit{The expression} \( \frac{0}{0} \) \textit{is undefined because it is indeterminate: it does not correspond to a unique real number.}

\section*{Proof}

Assume, for contradiction, that there exists a real number \( c \in \mathbb{R} \) such that:
\[
\frac{0}{0} = c
\]

By the definition of division, we know:
\begin{equation*}
\frac{a}{b} = c \iff a = b \cdot c, \quad \text{provided that } b \neq 0
\end{equation*}

Applying this to our assumption:
\begin{equation*}
\frac{0}{0} = c \Rightarrow 0 = 0 \cdot c
\end{equation*}

However, it is also true that:
\begin{equation*}
\forall c \in \mathbb{R},\quad 0 \cdot c = 0
\end{equation*}

Hence, the equation \( 0 = 0 \cdot c \) is satisfied by \textbf{every} real number \( c \). That is:
\begin{equation*}
\exists c_1, c_2 \in \mathbb{R},\quad c_1 \ne c_2,\quad \text{such that } 0 = 0 \cdot c_1 = 0 \cdot c_2
\Rightarrow \frac{0}{0} = c_1 \quad \text{and} \quad \frac{0}{0} = c_2
\end{equation*}

This contradicts the requirement that the result of division be \textbf{unique}.

\medskip

\noindent Therefore, the assumption that \( \frac{0}{0} \) equals any specific real number leads to contradiction.

\section*{Conclusion}

\begin{equation*}
\boxed{
\frac{0}{0} \text{ is undefined because it does not yield a unique real value; it is indeterminate.}
}
\end{equation*}

\end{document}
